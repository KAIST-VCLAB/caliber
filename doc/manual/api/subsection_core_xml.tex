\subsection{\texttt{caliber.io.xml}: XML input}
\subsubsection{XML file structure}
All items are elements.
\begin{itemize}
	\item \texttt{nodes}: A list of nodes.
	\begin{itemize}
		\item \texttt{node}: Information about a single node.
		\begin{itemize}
			\item \texttt{name}, \texttt{parent}, \texttt{K}, \texttt{kc}:
				As arguments to \texttt{InitOpt.addNode}. See below for how to specify a matrix.
				\texttt{parent} is optional.
			\item \texttt{Ms}: A list of matrices (again, see below), 
				\texttt{unknown} empty elements, or \texttt{unknowns} elements whose content
				specifies how many unknown states are to be added.
			\item \texttt{node}: nodes can be nested; the default parent of a node is 
				its parent in the XML, or the root if it is at the top level.
		\end{itemize}
	\end{itemize}
	\item \texttt{observations}: A list of observations.
	\begin{itemize}
		\item \texttt{observation}: Information about a single observation.
		\begin{itemize}
			\item \texttt{camera}, \texttt{pointset}, \texttt{imagePoints}, \texttt{worldPoints},
				\texttt{Q}, \texttt{weight}:
				As arguments to \texttt{InitOpt.addObservation}. See below for how to specify a matrix.
				\texttt{parent} is optional.
			\item \texttt{states}: A mapping of nodes to states. See below for details.
		\end{itemize}
	\end{itemize}
\end{itemize}

\paragraph{Specifying matrices.} Specify a matrix as in this example:
\begin{verbatim}
<matrix>
    <mr><md>1</md><md>0</md><md>0</md><md>0</md></mr>
    <mr><md>0</md><md>1</md><md>0</md><md>0</md></mr>
    <mr><md>0</md><md>0</md><md>1</md><md>0</md></mr>
    <mr><md>0</md><md>0</md><md>0</md><md>1</md></mr>
</matrix>
\end{verbatim}

\paragraph{Specifying states.}
A \texttt{states} element consists of a list of node states:
\begin{itemize}
	\item \texttt{state}: A mapping for a single node.
	\begin{itemize}
		\item \texttt{node}, \texttt{stateIndex}: A node name and its corresponding 
		state index.
	\end{itemize}
\end{itemize}

\subsubsection{\texttt{readXML}: reads a XML file into an \texttt{InitOpt} instance}
\texttt{readXML(initOpt, xmlFile)}
\paragraph{Overview.}
This reads an XML file into an existing \texttt{InitOpt} instance.
\paragraph{Inputs.}
\begin{itemize}
	\item \texttt{initOpt}: The \texttt{InitOpt} instance to read the file into.
	\item \texttt{xmlFile}: The name of the XML file to read in.
\end{itemize}
\paragraph{Outputs.}
None.
\paragraph{Remarks.}
\subsubsection{\texttt{solveXML}: reads a XML file and solves it}
\texttt{optimizer = solveXML(xmlFile)}
\paragraph{Overview.}
This creates a new \texttt{InitOpt} instance, reads in the specified file, solves it, and returns the result.
\paragraph{Inputs.}
\begin{itemize}
	\item \texttt{xmlFile}: The name of the XML file to solve.
\end{itemize}
\paragraph{Outputs.}
As \texttt{caliber.initopt.InitOpt.solve()}.
\paragraph{Remarks.}