\section{Nodes}
\label{sec:nodes}

\subsection{Provided node class}

\subsubsection{\texttt{GeneralNode}: standard 6-dimensional parameterization}

\texttt{node = GeneralNode( name, parentID, data, initialMs )}

\paragraph{Overview.} This node class represents a node with a list of transformations that
are each either completely known or completely unknown. It is appropriate for solving for 
general rigid transformations, such as robot arm links and rigid camera separations.

\paragraph{Inputs.}
\begin{itemize}
	\item \texttt{name}: The name of the node--this should be unique.
	\item \texttt{parentID}: Specifies the parent of the node. For the root node this should be empty.
	For other nodes, you can either refer to the parent by name or by handle.
	\item \texttt{data}: Data to associate with the node. This is a \texttt{struct} with matrices
	as fields.
	\item \texttt{initialMs}: A cell array of 4x4 transformation matrices representing the transformations
	associated with the states of this node. For an initially unknown transformation, use the empty matrix;
	this will be filled in by the initializer.
\end{itemize}
\paragraph{Remarks.}
\texttt{GeneralNode} parameterizes matrices relative to their initial value using 
\texttt{data.r} and \texttt{data.t}. Each of these is a $3 \times n$ matrix where each column represents
one state.

\subsection{Node methods}

If you wish to implement your own observation class, you will want to subclass \\
\texttt{caliber.node.Node}.
Then, you will need to implement several methods.

\subsubsection{Standard methods}

\paragraph{\texttt{dData = dData(tweak)}}

Does the same thing as \texttt{tweak.dValues()}.

\paragraph{\texttt{M = relativeM(stateIndices, targetNode, targetStateIndices)}}

Computes the relative transformation matrix between this node and another node.

\begin{itemize}
    \item \texttt{stateIndices}: The state indices of this node. 
        Generally you are calling this method on a camera node of an observation, so this argument should be \texttt{observation.cameraStateIndices}.
    \item \texttt{targetNode}: The target node. Usually this should be \texttt{observation.pointNode}.
    \item \texttt{targetStateIndices}: The state indices of the target node. Usually this should be \texttt{observation.pointStateIndices}.
\end{itemize}

\paragraph{\texttt{[dMs, tweakIndices] = dRelativeMs(stateIndices, targetNode, targetStateIndices, tweak)}}

This computes the derivative of M relative to the tweak parameters.

\begin{itemize}
    \item \texttt{stateIndices}: The state indices of this node. 
        Generally you are calling this method on a camera node of an observation, so this argument should be \texttt{observation.cameraStateIndices}.
    \item \texttt{targetNode}: The target node. Usually this should be \texttt{observation.pointNode}.
    \item \texttt{targetStateIndices}: The state indices of the target node. Usually this should be \texttt{observation.pointStateIndices}.
    \item \texttt{tweak}: The tweak in question.
\end{itemize}

\texttt{dMs} is a cell array and \texttt{tweakIndices} is a matrix, both with six elements. Each element \texttt{dMs\{j\}} gives the derivative of the 
transformation matrix between the two nodes with respect to the \texttt{tweakIndices(j)}th tweakable parameter of the tweak.
Elements of \texttt{dMs} may be empty matrices, indicating derivatives that are indentically zero; some \texttt{tweakIndices} may be
repeated, in which case the derivative is the sum of the corresponding elements of \texttt{dMs}.

\paragraph{\texttt{s = getStateString(stateIndices)}}

Gets a string representing the state of this node. May be useful for plot labels.
\begin{itemize}
    \item \texttt{stateIndices}: The state indices of this node. 
        Generally you are calling this method on a node of an observation, so this argument should be \texttt{observation.cameraStateIndices}
        or \texttt{observation.pointStateIndices}.
\end{itemize}

\subsubsection{Optional methods}
None currently.

\subsubsection{Required abstract methods}

\paragraph{\texttt{result = isKnown(obj, stateIndices)}}
Return true iff the transformation matrix is known for the specified state or all states.
If \texttt{stateIndices} has one element, return the answer for the \texttt{stateIndices}th element.
If \texttt{stateIndices} has more than one element, return the answer for the 
\texttt{stateIndices(obj.index)}th state.
(\texttt{obj.index} is automatically filled in when the object is assigned to a \texttt{Tree}).
If \texttt{stateIndices} is not given, return a logical array indicated which states have known 
transformation matrices.

\paragraph{\texttt{n = numberOfStates(obj)}}
Return the number of states associated with this node.

\paragraph{\texttt{M = getInitialM(obj, stateIndices)}}
Return the initial (i.e. pre-optimizer) transformation matrix of the \texttt{stateIndices(obj.index)}th state.

\paragraph{\texttt{setInitialM(obj, index, M)}}
Set the initial (i.e. pre-optimizer) transformation matrix of the \texttt{stateIndices(obj.index)}th state.
This is used by the initializer to set estimates for initially unknown transformations.

\paragraph{\texttt{M = M(obj, stateIndices)}}
Set the ``net'' transformation matrix (i.e. after parameters have been modified) of the \texttt{stateIndices(obj.index)}th state.

\paragraph{\texttt{[dMs, tweakIndices] = dMs(obj, state, tweak)}}
This should return the derivatives of the transformation matrix of this node for the given state
and the corresponding tweak indexes. You only need to return 
the derivatives that are nonzero as the cell array \texttt{dMs} and their corresponding tweak indexes 
as the array \texttt{tweakIndices}.
