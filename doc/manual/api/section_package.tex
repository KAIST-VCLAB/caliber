\section{Package structure}
For those of you interested in how the code is laid out, here is a summary of Caliber's package structure:

\subsection{Runnables}
These subpackages are the highest-level code and are generally run from the command line.
\subsubsection{\texttt{caliber.special}}
This package contains functions for calibrating particular types of configurations, such as rigid camera clusters.
\subsubsection{\texttt{test} subpackages}
Subpackages named ``\texttt{test}'' generally contain Caliber scripts or unit tests. Most are out of date.

\subsection{Core logic}
These subpackages implement the core logic of caliber.
\subsubsection{\texttt{caliber.tree}}
This subpackage contains the \texttt{Tree} class, which defines the problem description.
\subsubsection{\texttt{caliber.node}}
This subpackage contains the abstract \texttt{Node} class and its default implementation \texttt{GeneralNode}.
\subsubsection{\texttt{caliber.observation}}
This subpackage contains the abstract \texttt{Observation} class and some subclasses.
\subsubsection{\texttt{caliber.init}}
This subpackage contains the initializer code.
\subsubsection{\texttt{caliber.opt}}
This subpackage contains the optimizer code.

\subsection{Helper functions}
These packages contain mostly small functions that are used in other parts of caliber.
\subsubsection{\texttt{caliber.io}}
Functions here help process input to Caliber, or produce non-graphical output.
\subsubsection{\texttt{caliber.math}}
Various mathematical functions.
\subsubsection{\texttt{caliber.plot}}
Functions here deal with graphical output.
