\section{Observations} \label{sec:observations}

\subsection{Provided observation classes}

A few observation classes are provided for conventional cameras.

\subsubsection{\texttt{IndependentObservation}: Observation of known 3-D points}

\texttt{observation = IndependentObservation( cameraName, pointName, imagePoints, worldPoints, Q, cameraStates, pointStates )}

\paragraph{Overview.} This observation class represents observations of 3-D points whose positions in the 
pointset frame are assumed to be perfectly known. It is appropriate for Bouguet-style observations.

\paragraph{Inputs.}
\begin{itemize}
	\item \texttt{cameraName}: The name of the camera node.
	\item \texttt{pointName}: The name of the pointset node.
    \item \texttt{imagePoints}: The observed image points, as a 2 x $n$ matrix.
    \item \texttt{worldPoints}: The corresponding world points (in the pointset node frame), as a 3 x $n$ matrix.
    \item \texttt{Q}: An estimate for the relative transformation from the camera node to the pointset node.
    \item \texttt{cameraStates}: A \texttt{containers.Map} mapping node names to state indices for the camera node. 
        Nodes with no explicitly specified state are assumed to be in their first state.
    \item \texttt{pointStates}: Optional. A \texttt{containers.Map} mapping node names to state indices for the pointset node. 
        Defaults to the same as \texttt{pointStates}.
\end{itemize}
\paragraph{Remarks.}

\subsubsection{\texttt{NodeObservation}: Observation of points assoicated with a node}

\texttt{observation = NodeObservation(cameraName, pointName, imagePoints, pointIndices, Q, cameraStates, pointStates) }

\paragraph{Overview.} This observation class represents observations of 3-D points whose positions 
may be shared between multiple observations and may not be perfectly known. 
This may be useful if you have point correspondences from a structure-from-motion algorithm.

\paragraph{Inputs.}
\begin{itemize}
	\item \texttt{cameraName}: The name of the camera node.
	\item \texttt{pointName}: The name of the pointset node. The pointset node should have a field \texttt{points} 
        in its \texttt{data} struct representing world points in that node's frame as a 3 x $n_\mathrm{max}$ matrix.
    \item \texttt{imagePoints}: The observed image points, as a 2 x $n$ matrix.
    \item \texttt{pointIndices}: A list of indices, which select $n$ columns of \texttt{pointNode.data.points} that 
        correspond to observed points.
    \item \texttt{Q}: An estimate for the relative transformation from the camera node to the pointset node.
    \item \texttt{cameraStates}: A \texttt{containers.Map} mapping node names to state indices for the camera node. 
        Nodes with no explicitly specified state are assumed to be in their first state.
    \item \texttt{pointStates}: Optional. A \texttt{containers.Map} mapping node names to state indices for the pointset node. 
        Defaults to the same as \texttt{pointStates}.
\end{itemize}
\paragraph{Remarks.}

\subsection{Implementing observation classes}

If you wish to implement your own observation class, you will want to subclass \\
\texttt{caliber.observation.Observation}.
Then, you will need to implement several methods.

\subsubsection{Optional methods}

\paragraph{\texttt{preSolve(obj, optimizer)}}

This is called before the optimizer calls \texttt{lsqnonlin}. 
This is a good place to do any precomputation or caching that affects the entire optimzation.
The default behavior is to query the optimizer for handles to the nodes with names
\texttt{obj.cameraNodeName} and \texttt{obj.pointNodeName},
and store them in \texttt{obj.cameraNode} and \texttt{obj.pointNode},
and to use \texttt{obj.cameraStates} and \texttt{obj.pointStates} to compute a vector 
that maps node indices to state indices in \texttt{obj.cameraStateIndices}
and \texttt{obj.pointStateIndices}. 
You will probably want to call \texttt{preSolve@super(obj, optimizer)}
if you override this method.

\paragraph{\texttt{plotImagePoints(obj)}}

If you implement this function, it should generate a plot of the image points for this observation.

\paragraph{\texttt{drawExtrinsics(obj, scale)}}

If you implement this function, it should draw a 3-D representation of the observation's extrinsics onto the current plot.
For example, you might draw three lines for the camera and pointset node frames representing their axes.
\texttt{scale} gives an idea of how large this representation should be.

\paragraph{\texttt{preIteration(obj)}}

This is called before each iteration of the optimizer, after all node data has been updated. This is a good place to 
do any precomputation or caching that affects a single iteration.

\subsubsection{Required abstract methods}

\paragraph{Constructor.}

Not technically an abstract method but worth mentioning. You will want to at least set the following inherited members:
\begin{itemize}
    \item \texttt{obj.cameraNodeName}: The name of the camera node.
    \item \texttt{obj.pointNodeName}: The name of the pointset node.
    \item \texttt{obj.cameraStates}: A \texttt{containers.Map} mapping node names to state indices for the camera node. 
        Nodes with no explicitly specified state are assumed to be in their first state.
    \item \texttt{obj.pointStates}: A \texttt{containers.Map} mapping node names to point indices for the camera node. 
        Nodes with no explicitly specified state are assumed to be in their first state.
\end{itemize}

\paragraph{\texttt{Q = getQ(obj)}}

This should return an estimate of the relative transformation between the camera and the pointset. Often you will
compute this externally, pass the estimate as an argument into the observation's constructor, then simply return it for this method.

\paragraph{\texttt{imagePoints = projectedImage(obj)}}

Compute the reprojection for the observation.

\paragraph{\texttt{imageError = reprojectionError(obj)}}

Compute the reprojection error for the observation. The output should be a column vector.

\paragraph{\texttt{errorJacobian = reprojectionErrorJacobian(obj, tweak)}}

Compute the Jacobian of the reprojection error for the observation with respect to the given tweak.
The output should be a matrix, preferably sparse. 
The $i$th row of the matrix corresponds to the $i$th element of the value that 
\texttt{reprojectionError} returns. The $j$th column of the matrix corresponds to the $j$th
nonzero value of \texttt{tweak.paramIndices}, and thus the matrix has a number of columns
equal to \texttt{nnz(tweak.paramIndices)}. 

For example, suppose \texttt{tweak.node} matches the camera node of the observation;  
\texttt{tweak.paramName} is \texttt{'K'}, or the camera matrix;
and \texttt{tweak.paramIndices} is

\begin{center}\begin{tabular}{ccc}
    1 & 0 & 3 \\
    0 & 2 & 4 \\
    0 & 0 & 0
\end{tabular}\end{center}

corresponding to the focal length and principal point parameters

\begin{center}\begin{tabular}{ccc}
    $f_x$ & - & $p_x$ \\
    - & $f_y$ & $p_y$ \\
    - & - & -
\end{tabular}\end{center}

Then the Jacobian for this tweak is

\begin{center}\begin{tabular}{c|cccc}
    & $f_x$ & $f_y$ & $p_x$ & $p_y$ \\
	\hline
	$x_1$ & $\frac{\partial x_1}{\partial f_x}$ & $\frac{\partial x_1}{\partial f_y}$ & $\frac{\partial x_1}{\partial p_x}$ & $\frac{\partial x_1}{\partial p_y}$ \\
	$y_1$ & $\frac{\partial y_1}{\partial f_x}$ & $\frac{\partial y_1}{\partial f_y}$ & $\frac{\partial y_1}{\partial p_x}$ & $\frac{\partial y_1}{\partial p_y}$ \\
	$x_2$ & $\frac{\partial x_2}{\partial f_x}$ & $\frac{\partial x_2}{\partial f_y}$ & $\frac{\partial x_2}{\partial p_x}$ & $\frac{\partial x_2}{\partial p_y}$ \\
	$y_2$ & $\frac{\partial y_2}{\partial f_x}$ & $\frac{\partial y_2}{\partial f_y}$ & $\frac{\partial y_2}{\partial p_x}$ & $\frac{\partial y_2}{\partial p_y}$ \\
	\vdots & \vdots & \vdots & \vdots & \vdots \\
	$x_n$ & $\frac{\partial x_n}{\partial f_x}$ & $\frac{\partial x_n}{\partial f_y}$ & $\frac{\partial x_n}{\partial p_x}$ & $\frac{\partial x_n}{\partial p_y}$ \\
	$y_n$ & $\frac{\partial y_n}{\partial f_x}$ & $\frac{\partial y_n}{\partial f_y}$ & $\frac{\partial y_n}{\partial p_x}$ & $\frac{\partial y_n}{\partial p_y}$
\end{tabular}\end{center}

where $\left( x_i, y_i \right)$ are the coordinates of the $i$th image point. Note that the order of the columns corresponds to the order of the nonzero
entries of \texttt{tweak.paramIndices}.

See Sections \ref{sec:tweaks} and \ref{sec:nodes} for details on accessing and computing 
tweak and node information.

\subsubsection{\texttt{StandardObservation}: conventional cameras}

If you are using a conventional camera, you may want to subclass \\
\texttt{caliber.observation.StandardObservation}
instead. This subclass implements most of the abstract methods of \\
\texttt{caliber.observation.Observation}, leaving
just the constructor and two new abstract functions to implement. 
Additionally, it caches the standard intrinsic and extrinsic parameters in \texttt{obj.M}, \texttt{obj.kc}, and \texttt{obj.K}
in \texttt{preIteration()} so you can look those up directly.

\paragraph{\texttt{points = getPoints(obj)}}

This method should return a set of 3-D points as a 3 x $n$ matrix.

\paragraph{\texttt{J = nonstandardJacobian(obj, tweak)}}

This method should compute the Jacobian of the image points with respect to a tweak
whose \texttt{paramName} is not one of \texttt{'K'}, \texttt{'kc'}, \texttt{'r'}, or \texttt{'t'}
(these cases are taken care of by \texttt{StandardObservation}).
