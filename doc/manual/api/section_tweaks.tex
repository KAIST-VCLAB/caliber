\section{Tweaks}
\label{sec:tweaks}

\subsection{Tweak specification}

A tweak specification consists of a cell array, where each row contains four elements
that specify one or more parameters that are allowed to vary during the optimization.
Parameters are stored as matrices, which are themselves stored as elements of a \texttt{struct}.
These are \texttt{struct}s are the \texttt{data} member of \texttt{Node} classes.

\subsubsection{Node}
The first element in each row specifies which node contains the parameter to be tweaked.
This can be the name of the node, or a handle to the node itself.

\subsubsection{Parameter name}
The second element in each row specifies which field(s) of \texttt{node.data} can be modified.
This can be a string to specify a single field, or a cell array of strings to specify multiple 
fields at once.

\subsubsection{Parameter index(es)}
The third and fourth element in each row specifies which elements of \texttt{node.data.(parameterName)}
are allowed to vary. If the fourth element is empty, only the third element is used; otherwise the 
third element specifies the row and the fourth element the column. Both standard and logical indexing
are permitted; if you use logical indexing, remember to cast the indexing array to a logical array first.
You can also use the string \texttt{'all'}, which will select all elements (along that direction).

\subsection{Tweak objects}
Caliber uses the tweak specification to construct a \texttt{Tweak} object.

\paragraph{\texttt{tweak.node}}

A handle to the node that this tweak refers to.

\paragraph{\texttt{tweak.paramName}}

The parameter name that the tweak refers to. In other words, the tweak will
affect elements of \texttt{tweak.node.data.(tweak.paramName)}.

\paragraph{\texttt{tweak.paramIndices}}

This is a matrix whose nonzero elements indicate values of \\
\texttt{tweak.node.data.(tweak.paramName)} 
which are allowed to vary during the optimization. The value of the $j$th such element is equal to $j$.
For example, a camera matrix \texttt{'K'} whose focal length and principal point is allowed to vary
would have \texttt{tweak.paramIndices} equal to

\begin{center}\begin{tabular}{ccc}
    1 & 0 & 3 \\
    0 & 2 & 4 \\
    0 & 0 & 0
\end{tabular}\end{center}

For the extrinsic parameters \texttt{'r'} and \texttt{'t'}, the \texttt{tweak.paramIndices} are of the form

\begin{center}\begin{tabular}{c|cccc}
	& \multicolumn{4}{c}{Node states} \\
	& 1 & 2 & $\cdots$ & m \\
	\hline
	$x$ & ? & ? & $\cdots$ & ? \\
	$y$ & ? & ? & $\cdots$ & ? \\
	$z$ & ? & ? & $\cdots$ & ?
\end{tabular}\end{center}

\paragraph{\texttt{dValues = tweak.dValues()}}

Returns a cell array. The $j$th element of the array gives the derivative of 
\texttt{tweak.node.data} with respect to the $j$th nonzero element of \texttt{tweak.paramIndices}.
This turns out to be equal to \texttt{tweak.paramIndices ==} $j$.
