\section{I/O}

\subsection{\texttt{caliber.io.bouguet}: Bouguet calibration input}
\texttt{[ imagePoints, worldPoints, Q, data ] = bouguet( resultfile, indices )}
\paragraph{Overview.} This function reads a \texttt{Calib\_results.mat} file generated
by a Bouguet calibration and extracts the parameters for a single image.
Apart from this data, you will also need a tree structure to fully define the problem
to solve.
\paragraph{Inputs.}
\begin{itemize}
	\item \texttt{resultfile}: The name of a \texttt{Calib\_results.mat} generated
		by Bouguet.
	\item \texttt{indices}: The indices of the images within the file whose parameters are
		to be retrieved.
\end{itemize}
\paragraph{Outputs.}
\begin{itemize}
	\item \texttt{imagePoints}: A cell vector of $2 \times n$ arrays of observed image point coordinates.
	\item \texttt{worldPoints}: A cell vector of $3 \times n$ arrays of corresponding worldspace coordinates,
		in the pointset's frame.
	\item \texttt{Q}: A cell vector of estimated transformations between the camera and pointset frames.
    \item \texttt{data}: A struct containing intrinsic information for the camera.
\end{itemize}
\paragraph{Remarks.} 
	At current Bouguet is the only input format that we have specific code for.
	Note that Bouguet's convention has $z$ pointing in the direction that 
	the camera is looking, $y$ pointing downwards, and $x$ to the right from the
	camera's point of view.