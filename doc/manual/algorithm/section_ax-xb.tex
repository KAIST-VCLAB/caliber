\section{AX = XB details}
\label{sec:ax-xb_details}

\subsection{Separation of rotation and translation}

It is sometimes useful to separate the $AX = XB$ equation on full rigid transformations into its translational
and rotational parts. This gives

\begin{align}
	R_A R_X = R_X R_B \label{eq:ax-xb_rotation} \\
	\left(R_A - I \right) t_x = R_X t_b - t_a \label{eq:ax-xb_translation}
\end{align}

\subsection{Screw axes}

It is also useful to observe that these equations describe a similarity transformation $B = X^{-1}AX$.
Furthermore, Chen \cite{Chen1991} gives a method of characterizing $AX = XB$ systems using 
screw theory. Briefly, screws are a way of representing rigid transformations. They consist of 
a screw axis, which is a line in space, a translation magnitude along that axis, and a 
rotation magnitude about that axis. Screws have the property that the translation and rotation magnitudes
are invariant with rigid transformations, whereas the screw axis transforms with rigid transformations.
Therefore, if we one or more $AX = XB$ equations, we can think of this problem as one where
we attempt to find the transformation $X$ that takes the screw axes given by $A_1, A_2, \ldots$ 
to those given by $B_1, B_2, \ldots$

\subsection{Cases with rotation}

\subsubsection{Nonparallel (skew or intersecting) screw axes (0 DoF)}

Park and Martin \cite{Park1994} give a method to compute the unique answer in the case where the 
screw axes are nonparallel and the rotation magnitudes are not near the identity or a 180-degree rotation 
(so the screw axes are well-defined). If the problem is overconstrained by having more than two $AX = XB$
equations, they give a least-square solution. In particular, they compute the rotational matrix that 
transforms a $3 \times n$ matrix of vectors $\alpha$ to another one $\beta$ as

\begin{align}
	M = \beta \alpha^T \\
	R = \left( M^T M \right)^{-\frac{1}{2}} M^T
\end{align}

\subsubsection{Parallel screw axes (1 DoF)}

Chen \cite{Chen1991} shows that in the case where the screw axes are parallel but not coincident, 
the rotation can be uniquely determined, but the translation has one degree of freedom.
For two such pairs of rigid transformations $A_1, B_1$ and $A_2, B_2$, 
we can construct an orthogonal using the (mutual)
screw axis direction, the direction separating the two screw axes, and their cross product.
We can then solve for the rotation as before. This is depicted in Figure \ref{fig:parallel-axis}.

\begin{figure}
\tdplotsetmaincoords{120}{60}
\begin{center}
\begin{tikzpicture}[tdplot_main_coords]
	\draw[->] (0,0,0) -- (0,0,4) node[anchor=north west]{$A_1$};
	\draw[->] (2,0,0) -- (2,0,4) node[anchor=north west]{$A_2$};
	\draw (0,0,1.5) -- (0.5,0,1.5);
	\draw (0.5,0,1) -- (0.5,0,1.5);
	\draw (0,0,1.5) -- (0,-0.5,1.5);
	\draw (0,-0.5,1) -- (0,-0.5,1.5);
	\draw (0.5,0,1) -- (0.5,-0.5,1);
	\draw (0,-0.5,1) -- (0.5,-0.5,1);
	\draw[dashed, ->] (0,0,1) -- (2,0,1);
	\draw[dashed, ->] (0,0,1) -- (0,-2,1);
	
	\tdplotsetrotatedcoords{40}{20}{70}
	\coordinate (Shift) at (5,5,0);
	\tdplotsetrotatedcoordsorigin{(Shift)}
	
	\draw[thick, shorten >=1cm,shorten <=2cm, ->] (0,0,1) -- (5,5,1) node[midway, above] {X};
	
	\draw[tdplot_rotated_coords, ->] (0,0,0) -- (0,0,4) node[anchor=north west]{$B_1$};
	\draw[tdplot_rotated_coords, ->] (2,0,0) -- (2,0,4) node[anchor=north west]{$B_2$};
	\draw[tdplot_rotated_coords] (0,0,1.5) -- (0.5,0,1.5);
	\draw[tdplot_rotated_coords] (0.5,0,1) -- (0.5,0,1.5);
	\draw[tdplot_rotated_coords] (0,0,1.5) -- (0,-0.5,1.5);
	\draw[tdplot_rotated_coords] (0,-0.5,1) -- (0,-0.5,1.5);
	\draw[tdplot_rotated_coords] (0.5,0,1) -- (0.5,-0.5,1);
	\draw[tdplot_rotated_coords] (0,-0.5,1) -- (0.5,-0.5,1);
	\draw[tdplot_rotated_coords, dashed, ->] (0,0,1) -- (2,0,1);
	\draw[tdplot_rotated_coords, dashed, ->] (0,0,1) -- (0,-2,1);
\end{tikzpicture}
\end{center}
\caption{Parallel screw axis frames.}
\label{fig:parallel-axis}
\end{figure}


Since $R_A$ always has at least one eigenvalue of unity, 
$\left(R_A - I \right)$ in the translation equation \ref{eq:ax-xb_translation} 
has a one-dimensional null space, and the solution space has one degree of freedom.
Since all the screws are parallel, this null space is the same for all of the equations.
In this case, we attempt to keep the translation as close to the origin as possible,
which we do by adding another row to the matrix $r_a' t_x = 0$.
Finally, we stack the equations produced by each pair of $AX = XB$ equations to produce a single
least-squares solution for $t_x$.

Note that this gives an alternate method for solving the skew screw axis case: choose one screw axis
as one direction, the direction along the shortest line between two screw axes as another, and their 
cross product as the third. Since in this case the null space of $\left(R_A - I \right)$ for
different equations are not the same, the translation has a unique solution.

\subsubsection{Coincident screw axes (2 DoF)}

In the coincident screw axis case, only one direction for the $A$s and for the $B$s can be determined;
in this case we choose the rotation axis to be along the (average) cross product of the screw axes,
which minimizes the rotation angle. We then solve for the translation as per the parallel axis case.

\subsection{Cases without rotation}

In some cases we may not have rotations at all; here the screw axis has direction equal to the translation
but its offset perpendicular to the axis direction is undefined. 
The rotation equation \label{eq:ax-xb_rotation} reduces to $R_X = R_X$, which gives us no information. 
The left side of the translation equation \ref{eq:ax-xb_translation} is zero, which means that $t_x$
cannot be determined at all, but there may be some information about $R_X$.

\subsubsection{Nonparallel translation (3 DoF)}

If the translations are nonparallel, we can use two non-parallel translations and their cross product to
determine the rotation uniquely, as in the noncoincident screw axis cases. 
As noted above, the translation is completely undetermined.

\subsubsection{Parallel translation case (4 DoF)}

Similarly, if we have only parallel translations, there is one degree of freedom. 
We can use the same method as the coincident screw axis case to determine a feasible rotation.
Again the translation is completely undetermined.

\subsubsection{No information case (6 DoF)}

If we have no information at all (no equations, or all $A$s and $B$s are the identity), then we
simply guess $X$ to be the identity.

\subsection{Omitted cases}

There are some cases which we have omitted here:

\subsubsection{180-degree rotations}

Chen \cite{Chen1991} addresses the case where some or all of the rotations are 180 degrees.
In these cases the rotation axis is defined, but only up to a binary choice between a direction
and its reverse.

\subsubsection{Mixed cases}

Chen \cite{Chen1991} also addresses the cases where different equations fall into different cases;
for example, one equation with rotation and one equation with only rotation. These cases
are amenable to the previous sort of analysis.

\subsubsection{Practical use}

We have not covered these cases so far because they are rare in practical use. Typically, the $A$s and $B$s
in the equations result from the relationship between two positions of a joint; this means 
skew screw axes for a ball joint (or equivalent, such as a spherical gantry), 
coincident screw axes for a revolute or cylindrical joint, and parallel translation for a prismatic joint;
and the changes in the estimated transformations from the corresponding observations.
Therefore, it is rare to find 180-degree rotations or mixed types of equations.
However, they could be added without too much difficulty.
