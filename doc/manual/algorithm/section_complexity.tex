\section{Initializer complexity}

\subsection{Binary choice via XAX = B}

The core gadget relies on cycles of the form $XAX = B$.

\subsubsection{Background}

\paragraph{Translation given rotations.}

Suppose we know the rotational part of $X$ but not the translational part--denote these by $R_X$ and $t_x$.
Then solving for the unknown $t_x$ reduces to
\begin{align}
	R_X R_A t_x + R_X t_a + t_x = t_b \\
	\left( R_X R_A + I \right) t_x = t_b - R_X t_a
\end{align} 
This has a unique solution as long as $R_X R_A + I$ is invertible. To see when this is the case,
consider this product for a unit vector $x$:
\begin{align}
	x^{-1} \left( R_X R_A + I \right) x = x^{-1} R_X R_A x + 1
\end{align}
For any rotation $R$, $x^{-1} R x$ is only $-1$ when $R$ is a 180-degree rotation.
Otherwise this shows that $R_X R_A + I$ is positive-definite and therefore invertible.
Thus it can be said that unless $R_X R_A$ is a 180-degree rotation, $t_x$ has a unique solution
given $R_X$.

\paragraph{Rotations.}
Inspiration taken from \\
\texttt{http://www.mathworks.com/matlabcentral/newsreader/view\_thread/165797}

From a single equation we have
\begin{align}
	XAX = B \\
	AXAX = AB \\
	AX = \left( AB \right)^\frac{1}{2}
\end{align}
Suppose we consider only the rotational part. 
The square root has multiple values; however, since our solution is restricted to 
$SO \left( 3 \right)$, this limits the possibilities.

\paragraph{The (more) undetermined case.}
If $R_A R_B = I$, then $I$ and any 180-degree rotation is a possible
square root. In our two-identical-tablet case, this would correspond to a situation 
where the two tablets have exactly anti-parallel camera axes.

\paragraph{The common case.}
Otherwise, $R_A R_B$ has a defined axis of rotation, 
and $\left( R_A R_B \right)^\frac{1}{2}$ must share that same axis, 
since applying any rotation twice preserves the axis of 
rotation. 
The magnitude of the rotation must be half that of $R_A R_B$,
plus an integer multiple of $\pi$. Since a difference of $2 \pi$ leaves the rotation matrix unchanged,
this produces two possible values for $\left( R_A R_B \right)^\frac{1}{2}$ and therefore two possible 
values for $R_X$.

\subsubsection{A special case}

\label{sec:special}
Let there exist some label $X$. Suppose we have a cycle that produces an $XAX = B$
equation. Let the rotational parts of $A$ and $B$ equal
\begin{align}
	R_A = R_B = \left[\begin{array}{ccc}
		0 &  -1 & 0 \\
		1 & 0 & 0 \\
		0 & 0 & 1
	\end{array} \right]
\end{align}

From this we have

\begin{align}
	R_X \in \left\{
		I,
		I'
	\right\} \\
	I = \left[\begin{array}{ccc}
			1 &  0 & 0 \\
			0 & 1 & 0 \\
			0 & 0 & 1
		\end{array} \right] \\
	I' = \left[\begin{array}{ccc}
			-1 &  0 & 0 \\
			0 & -1 & 0 \\
			0 & 0 & 1
		\end{array} \right]
\end{align}

Now let us examine the translational parts of $A$, $B$, and $X$. Suppose

\begin{align}
	t_a = \left[\begin{array}{c}
			0 \\ 0 \\ 0
		\end{array} \right] \\
	t_b = \left[\begin{array}{c}
			b \\ b \\ 0
		\end{array} \right]
\end{align}

for some constant $b$. In the case where $R_X = I$, we have

\begin{align}
	\left( R_X R_A + I \right) t_x = t_b - R_X t_a \\
	\left( R_A + I \right) t_x = t_b \\
	\left[\begin{array}{ccc}
		1 &  -1 & 0 \\
		1 & 1 & 0 \\
		0 & 0 & 1
	\end{array} \right] 
	t_x = \left[\begin{array}{c}
			b \\ b \\ 0
		\end{array} \right] \\
	t_x = \left[\begin{array}{c}
			b \\ 0 \\ 0
		\end{array} \right]
\end{align} 

In the case where $R_X = I'$, we have

\begin{align}
	\left( R_X R_A + I \right) t_x = t_b \\
	\left[\begin{array}{ccc}
		1 &  1 & 0 \\
		-1 & 1 & 0 \\
		0 & 0 & 1
	\end{array} \right] 
	t_x = \left[\begin{array}{c}
			b \\ b \\ 0
		\end{array} \right] \\
	t_x = \left[\begin{array}{c}
			0 \\ b \\ 0
		\end{array} \right]
\end{align}

Therefore, in this special case, $X$ either represents a step ``forwards'' of size $b$,
or a step to the left of size $b$ followed by an about-face. If $b = 0$, then the choice 
is between standing still and an about-face. Also note that if $b$ is an integer, all
arithmetic here is integer.

\subsection{Reduction from the (unique) partition problem}

We show a polynomial reduction from the (unique) partition problem to our initializer problem.

\subsubsection{The partition problem}

Recall the partition problem (PP). The definition of the problem is as follows:
given a set of positive integers $S = \left\{ x_1, x_2, \ldots x_n \right\}$, 
determine if there exists a subset of the integers $S_1$ such that
\begin{align}
	\sum_{x \in S_1} x = \sum_{x \not \in S_1} x
\end{align}

The standard form of the problem is well-known to be NP-complete; 
Dieudonne, et al. \cite{Dieudonne2010} show that, given an instance of this problem 
for which it is guaranteed that at least one solution exists, it is NP-hard
to determine whether there exists more than one solution.
They call this version the unique partition problem (UPP).

\subsubsection{The labeled transformation graph problem}

Now consider our initializer problem, which we will call the 
labeled transformation graph problem (LTGP).
The uniqueness version (ULTGP) of this problem asks whether, given an instance for LGTP which a solution
is guaranteed to exist, there exists more than one solution. (We consider swapping a particular choice
of $S_1$ with its complement in $S$ to be the same solution.)

\subsubsection{The core gadget}

Towards reducing PP to LTGP, we introduce the following gadget. For each $x_i \in S$, we introduce
two labels $X_i$ and $X_i'$. For both of these we construct an $XAX = B$ cycle of the form
described in Subsection \ref{sec:special}. For the one with $X_i$, we set $b = x_i$; for the one
with $X_i'$, we set $b = 0$. This produces four options for the product $X_i X_i'$, corresponding to 
our choice of picking $R_X = I$ or $R_X = I'$ for each rotational part:

\label{sec:steps}
\begin{enumerate}
	\item A step forward of size $x_i$ (choose $R_{X_i} = I$ and $R_{X_i'} = I$).
	\item A step forward of size $x_i$ followed by an about-face 
		(choose $R_{X_i} = I$ and $R_{X_i'} = I'$).
	\item A step left of size $x_i$ (choose $R_{X_i} = I'$ and $R_{X_i'} = I'$).
	\item A step left of size $x_i$ followed by an about-face (choose $R_{X_i} = I'$ and $R_{X_i'} = I$).
\end{enumerate}

We put one vertex of each such cycle at the root, so these are the only choices to be made here.
Since we do constant work per element of $S$, this stage is linear-time.

\subsubsection{A Manhattan walk}

With this set of gadgets in place, we introduce one more cycle, which we will treat as 
two paths from the root to a destination. On one side we constrain the transformation to the destination
to equal

\begin{align}
	T &= \left[\begin{array}{cccc}
			1 & 0 & 0 & \frac{1}{2} s \\
			0 & 1 & 0 & \frac{1}{2} s \\
			0 & 0 & 1 & 0 \\
			0 & 0 & 0 & 1
		\end{array} \right] \\
	\text{where} \ s &= \sum_i^n x_i
\end{align}

On the other side we have a path of $2n$ edges labeled such that they form the product 
\begin{align}
	T = \prod_i^n X_i X_i' = X_1 X_1' X_2 X_2' \ldots X_n X_n'
\end{align}

Again this is a linear-time transformation.

At each two-edge step towards the destination $X_i X_i'$ 
a solution must take one of the four options listed in Subsubsection \ref{sec:steps}.
This constrains our translations to lie only on cardinal directions. Since the Manhattan distance
to our target is $s$ and we only have a total of $s$ distance in translations in our product,
this means we must step towards the destination in every step. If we do an about-face 
before the last step, we will have to step away from the destination on the next step, and if 
we do an about-face on the last step, we will be facing the wrong direction. Therefore any solution must 
consist of only options 1 and 3. Since the forward steps total the same as the left steps in any solution,
this forms a one-to-one correspondence to a solution to PP--the indices of the steps in which we chose to move
forward correspond to the indices of the integers in the set we select to be in our partition.
Therefore PP reduces to LTGP, UPP reduces to ULTGP, and the initializer problem and its uniqueness version
are NP-hard.

